\documentclass[a4paper]{article}
\usepackage[utf8]{inputenc}
\usepackage{amsmath}
\usepackage{siunitx}
\usepackage[ngerman]{babel}
\usepackage{hyperref}
\usepackage{amssymb}
\usepackage[section]{placeins}


\title{Übungsblatt 1}

\author{Luca Krüger \\ \href{mailto:luca.krueger@uni-ulm.de}{luca.krueger@uni-ulm.de}}
\date{\today}

\addtocounter{section}{2}


\renewcommand{\thesubsection}{\alph{subsection})}

\begin{document}
\maketitle
\section*{Aufgabe \thesection}
	
	\subsection{}
	
	Eine Liste kann einfach mit den mitteln der Klasse Comparable sortiert werden. Alternativ kann auch ein Sortier-Operator übergeben werden, nach dem sortiert werden soll.\\
	Collection (ohne s) ist ein Interface und kein Datentyp oder Objekt. Ein Interface bietet nur die Möglichkeit der klassenübergreifenden Strukturieung von Methoden, die implementiert werden sollen. 

	\subsection{c)}
	Die Testklasse aus den folgenden beiden Teilaufgaben wurde in das package von blatt$2\_1$ implementiert.
	
	

\end{document}