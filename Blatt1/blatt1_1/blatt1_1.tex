\documentclass[a4paper]{article}
\usepackage[utf8]{inputenc}
\usepackage{amsmath}
\usepackage{siunitx}
\usepackage[ngerman]{babel}
\usepackage{hyperref}
\usepackage{amssymb}
\usepackage[section]{placeins}


\title{Übungsblatt 1}

\author{Luca Krüger \\ \href{mailto:luca.krueger@uni-ulm.de}{luca.krueger@uni-ulm.de}}
\date{\today}

\addtocounter{section}{1}


\renewcommand{\thesubsection}{\alph{subsection})}

\begin{document}
\maketitle
\section*{Aufgabe \thesection}
	
	\subsection{}
	
	\begin{table}[!h]
		\centering
		\begin{tabular}{l c c c}
			&hungry&food&legs \\ \hline
			Class Animal&\checkmark&\checkmark& \\
			Class Bird&&\checkmark&\checkmark \\
			Class Eagle&&\checkmark&\checkmark \\
		\end{tabular}
		\caption{Messtabelle für Versuch 1}
		\label{tab:messtab1}
	\end{table}

	\subsection{}
	
	Overloading findet in der Methode "fly" in der Klasse "Bird" statt. Die beiden Methoden unterscheiden sich in diesem Fall durch die Anzahl der übergebenen Parameter.\\
	Overriding der Funktion eat in der Klasse Bird.
	
	\subsection{}
	
	\begin{enumerate}
		\item erlaubt.
		\item erlaubt. Rückgabewert "32".
		\item nicht erlaubt. Typecasts funktionieren höchstens zwischen Klassen, die voneinander erben.
		\item erlaubt, wobei ein Typecast an dieser Stelle noch nicht einmal notwendig ist.
		\item erlaubt. Rückgabewert "32".
		\item erlaubt.
		\item nicht erlaubt. fly ist private deklariert.
		
	\end{enumerate}
	
	

\end{document}